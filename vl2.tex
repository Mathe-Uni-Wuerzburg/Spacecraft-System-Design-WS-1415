\begin{chapter}{13.10.2014}
 \[ m \cdot \ddot{\vec{r}} = F = - G \cdot \frac{m_1m_2}{r^2} \]
 $G$ ist die \indexx{universal gravity constant}: $6.674 \cdot 10^{-11} \frac{Nm^2}{kg^2}$
 
 \[ \ddot{\vec{r}} = \frac{\mu}{\|r\|^2} \]
 \begin{tabular}{c|l|c}
 $\mu$ & gravity constant of the specific gravity attractor.&\\
 $\mu_E$ & \indexx{gravity constant of earth}: & $398600 \frac{km^3}{s^2}$\\
 $m_E$ & \indexx{mass of earth}: &$5.97219 \cdot 10^{24} kg$\\
 \end{tabular}
 
 Die Lösungen dieser Gleichung sind Kegelschnitte.
 Außerdem gilt: $\mu_E = G\cdot m_E$\\

 
 \[ r = \frac{p}{1+\varepsilon\cdot\cos\varphi} \]
 \begin{tabular}{c|l|c}
 $r$ & radius towards focal point&\\
 $\varepsilon$ &eccentricity of the orbit&\\
 $p$ &parameters (spartial extensions of orbit)&\\
 $\varphi$ &true anomaly&\\
 \end{tabular}

 
 \textbf{Interrelationships between parameters:}\\
 $r_a$ apocenter, $r_p$ pericenter\\
 \[ a = \frac{p}{1-\varepsilon^2} \Rightarrow p = a(1-\varepsilon^2)= r_p(1+\varepsilon) = r_a(1-\varepsilon)\]
 \[ a = \frac{r_a+r_p}{2}\text{, } r_a = a(1+\varepsilon)\text{, }r_p=a(1-\varepsilon)\]
 \[ \varepsilon = \frac{r_a-r_p}{r_a+r_p} \]
 
 For the specific case of circular orbits:
 \begin{itemize}
  \item gravitational acceleration in distance $a$ to gravitational center: $\frac{\mu}{a^2}$
  \item centrifugal acceleration: $\omega^2\cdot r$ with $\omega = \frac{2\pi}{T}$ angular velocity and $T$ the orbital period
  \[ \left(\frac{2\pi}{T}\right)^2\cdot a \Rightarrow \frac{\mu}{a^2} = \left(\frac{2\pi}{T}\right)^2\cdot a \Rightarrow T = 2\pi\sqrt{\frac{a^3}{\mu}}\]
  which also holds for general ellipses
  \[v_\text{circle} = \frac{2\pi a}{T} = \sqrt{\frac{a^3}{\mu}} \]
 \end{itemize}

 \indexx{Potential energy}: $E_\text{pot} = \frac{-\mu m}{r} +C$\\
 \indexx{Kinetic energy}: $E_\text{kin} = \frac{m v^2}{2}$
 \[ E = E_\text{kin} + E_\text{pot} = \frac{mv^2}{2} - \frac{\mu m}{r} \stackrel{\substack{\varepsilon \rightarrow 1\text{, }\\v_a \rightarrow 0\text{, }\\r_a \rightarrow 2a}}{=} -\frac{\mu m}{2a} \text{(constant)}\]
 $v_a$ velocity at apocenter
 
 \indexx{Binet's equation}, \indexx{Vis-Viva equation}:
 \[ v = \sqrt{\mu \left( \frac{2}{r} - \frac{1}{a} \right)} \]

 \textbf{For circles:} $r=a$, $v=\sqrt{\frac{\mu}{a}}$\\
 $v$ \indexx{first cosmic velocity}: needed to bring a satellite on closed orbit(without perturbing forces, just gravity considered). In case of earth: $r = 6378km$, $\mu_E = 398600 \frac{km^3}{s^2}$\\
 $\Rightarrow 7.905 \frac{km}{s}$ required as minimum velocity from the rocket launch to insert into an earth orbit.
 
 A rocket launch in equatorial direction and in rotation direction of earth, we recieve a velocity component $0.463 \frac{km}{s}$ for free earth rotation rate.
 
 \textbf{For parabola:} $a\rightarrow \infty$, $v = \sqrt{\frac{2\mu}{r}}$
 $v$ \indexx{second cosmic velocity (escape velocity)}: is the minimum velocity applied to leave the gravitational field of the home planet. For earth: $11.179 \frac{km}{s}$

 Derive position of the satellite in the orbit plane, e.g. we want the true anomaly $\varphi$ as the function of time $\varphi = f(t)$. There is no explicit solution but only an algorithm using the support variable eccentric anomaly $E$ can be derived.
 
 \textbf{\indexx{Keplers equation}}:
 \[ E - \varepsilon\cdot\sin(E) = \frac{2\pi}{T}(t-t_\text{perigee}) =M(t) \]
 $M(t)$ average anomaly at time $t$, can be solved by numerical methods. From $E$, you can calculate $r$ and $\varphi$ as follows:\\
 \[r\sin\varphi = a\sin E \cdot \sqrt{1-\varepsilon^2}\]
 \[r\cos\varphi = a(\cos E - \varepsilon)\]
 \[\Rightarrow \tan \frac{\varphi}{2} = \sqrt{\frac{1+\varepsilon}{1-\varepsilon}}\cdot \tan\frac{E}{2}\]
 \[r=a(1-\varepsilon\cos E)\]
 
 \textbf{Satellite Ground Tracks:}\\
 Analyse the triangle KFB in spherical coordinates.
 
 \begin{tabular}{c|l|c}
 $\theta$ &\indexx{latitude}: &$\sin \theta = \sin i \sin (\omega + \varphi)$\\
 $\lambda$ &\indexx{longitude}: &$\cos \lambda_1 = \frac{\cos(\omega + \varphi)}{\cos \theta}$\\
 \end{tabular}

 \[ \lambda = \lambda_1 + \Omega + \lambda_G \]
 \[ \lambda_G = \lambda_{G0} - 0.25068448\degree /\text{min} \cdot t \text{~~~~~(Greenwich position versus $\gamma$)}\]
 Draw the line from the satellite to earth center, the intersection with surface is specified by $\theta$ and $\lambda$ (\indexx{subsatellite point}).
 
 \textbf{\indexx{Perturbations}}:\\
 \textbf{gravitational potential of earth}\\
 \[ u(r,\phi,\Lambda) \]
 $u$ gravitational at distance $r$ and latitude $\phi$ and longitude $\Lambda$ of the position.
 
 \[ u(r,\phi,\Lambda) = \frac{\mu}{r} \bigg( 1+ \sum_{n=2}^{\infty}\left[ \left(\frac{R_E}{r}\right)^n \cdot J_2 \cdot P_{n0} \cos\phi + \sum_{m=1}^{n} \left(\frac{R_e}{r}\right)^n \cdot \left( C_{nm} \cos(m\Lambda) + S_{nm} \sin (m\Lambda)\right)\cdot P_{nm} (\cos \phi) \right] \bigg) \]
 \indexx{spherical harmonic expansion} with:\\
 \begin{tabular}{c|l|c}
 $r$ &distance from the earth center&\\
 $P_{nm}$ &Legendre polynomials&\\
 $R_E$ &radius of earth &$6378 km$\\
 \end{tabular}
 
 $J_n$, $C_{nm}$, $S_{nm}$ are constants of body reflecting the mass distribution:\\
 $J_n$ zonal harmonic coefficient (mass distribution independent from longitude)\\
 $C_{nm}$, $S_{nm}$ earth's tesseral harminic coefficients (for $n \neq m$) or earth's sectoral harmonic coefficients (for $n = m$)

 
 By far, the largest coefficient is $J_n$ (related to equatorial bulge) by $3$ orders of magnitude.\\
 Example: Molniya orbits of $63,4\degree$ have minimum change in position of perigee and apogee.
 
 \textbf{Additional gravitational fields:}\\
 by other bodies in the solar system
 \[ a_d = \mu_d \sqrt{\vec{R}\cdot\vec{R}} \]
 with $\vec{R} = \frac{\vec{r}_{sd}}{\|\vec{r}_d\|}$
 \[ \frac{a_d}{a_c} = \frac{M_d}{M_c} \left( \frac{\|\vec{r}_s\|}{\| \vec{r}_{sd} \|} \right)^3 \sqrt{1+3\cos^2 \beta} = 2 \frac{M_d}{M_c} \left( \frac{\|\vec{r}_s\|}{\| \vec{r}_{d} \|} \right)^3\]
 \begin{tabular}{c|l|c}
  $a_c$ & actual acceleration &\\
 \end{tabular}
 
 \textbf{\indexx{atmospheric drag}:}\\
 Drag: $\vec{F}_0 = -\frac{1}{2}\rho S c_D \| \vec{v}_R \|^2 \frac{\vec{v}_r}{\| \vec{v}_r \|}$
 
 \begin{tabular}{c|l|c}
  $v_r$ & velocity vector of the spacecraft relative to the atmosphere& \\
  $\rho$ & atmospheric density & \\
  $S$& reference area of the spacecraft& \\
  $c_D$& drag coefficient of the spacecraft (determined in wind channel)& default: $c_d \approx 2.5$\\
 \end{tabular}
 
 Main effect of altitude changes and changes of orbital period.
 
 \textbf{\indexx{solar pressure}: }\\
 \begin{tabular}{c|l|c}
  $1AU$ & \indexx{distance earth-sun} & $149.6\cdot 10^6 km$\\
  & energy density & $1372 \frac{W}{m^2}$\\
 \end{tabular}

 from impulse conservation law:
 \[ p = \frac{1372 W}{c \cdot m^2} = \frac{1372 W~ s}{299793 km~m^2 } = 0.46 \cdot 10^{-5} \frac{N}{m^2}\]
 solar pressure is increased by reflection
 \[ p_\text{eff} = (1+r)p \]
 \begin{tabular}{c|l|c}
  $p_\text{eff}$ & effective pressure&\\
  $r$& reflectivity factor&\\
 \end{tabular}
 
 At equatorial orbits the solar pressure affects a change of eccentricity $\varepsilon$.
 
 \textbf{orbit change manouvers:}\\
 Based on impule conservation law.\\
 Propulsion systems:\\
 \begin{tabular}{rcl}
  Hydrazin & $2.2\frac{km}{2}$ & particle velocity\\
  2-component & $3\frac{km}{2}$ & particle velocity\\
  ion thrusters & $30\frac{km}{2}$ & particle velocity\\
 \end{tabular}
 \[ \Delta v_s m_s = -\Delta m_s v_p \]
 \begin{tabular}{c|l|c}
  $\Delta m_s$ &reduction of satellite mass due to ejected propellant&\\
  $v_p$ &velocity of propellant ejection&\\
 \end{tabular}
 \[ \Delta v_s = -\frac{\Delta m_s}{m_s}v_p \]
 \[ \int_{v_0}^v dv = -v_p \int_{m_s}^{m_s-m_p} \frac{dm}{m}\]
 \[ \Delta v_s = v_p (\log_2 m_s - \log_2 (m_s-m_p)) \]
 \indexx{Rocket equation}, \indexx{Ziolkowski equation}:
 \[ m_\text{after} = m_\text{before} \cdot e^{\frac{-\Delta v_s}{v_p}} \]
\end{chapter} 