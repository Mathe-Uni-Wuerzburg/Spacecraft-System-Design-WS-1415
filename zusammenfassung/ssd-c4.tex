\section{Distributed Satellite Systems}
\begin{itemize}
 \item \f{constellation}: similar trajectories without relative position control.
 \item \f{formation}: closed-loop onboard control for topology in the group.
 \item \f{swarm}: similar vehicles cooperating without fixed positions, selfdetermined.
 \item \f{cluster}: heterogenous system of vehicles for joint objective.
\end{itemize}
requirements on distributed satellite systems: coordination of
\begin{itemize}
 \item orbits at different altitudes
 \item optimal control strategies for position/attitude of components
 \item activities for heterogenous sensors
 \item information flow and storage
\end{itemize}
\f{Walker Delta Pattern Constellation}
\[ i: t/p/f \]
\begin{itemize}
 \item i: inclination
 \item t: total $\sharp$ satellites
 \item p: $\sharp$ equally shaped orbit planes
 \item f: relative phase difference between satellites in adjacent planes
\end{itemize}
Example: Galileo is $56\degree: 27/3/1$ with circular orbits ($h =23222km$), nine satellites always in view, one spare satellite in each plane.
\f{earth surface converage}
$ s = \frac{t}{p}$ number of satellites equally spaced in plane with angular distance $\Delta v = \frac{360\degree}{s}$. There are two cases:
\begin{itemize}
 \item $\Delta v < 2\cdot \lambda_\text{max} \Rightarrow$ area of continuous coverage exists (``street of coverage'')
 \item $\Delta v > 2\cdot \lambda_\text{max} \Rightarrow$ no street of coverage
\end{itemize}
Street-width: $\cos \lambda_\text{street} = \frac{\cos\lambda_\text{max}}{\cos\frac{\Delta v}{2}}$

\f{formation flying arcitectures and dynamics}
\begin{itemize}
 \item \f{virtual structure}: treated as single structure
 \item \f{behavioral strategies}: distributed control approach, following nature.
 \item \f{leader-follower}: divided into leaders and followers. followers track designated leaders with prescribed offset. absolute/relative control architecture.
\end{itemize}

\f{communication in low-earth orbit distributed satellite systems}
\begin{itemize}
 \item comm and tele-operation infrastructure is key element for distributed systems
 \item transfer position and observation data for formation flying
 \item amount of data increases with swarm size
 \item analyse pre-processing procedures, intersatellite links and ground station links
\end{itemize}

\f{conclusion on distributed satellite systems}
\begin{itemize}
 \item research field due to paradigm shift from one large spacecraft to several smaller crafts
 \item higher fault tolerance and robustness
 \item swarms are scaleable
 \item gun launches into orbit
 \item comination of big and small spacecrafts
 \item swarms for survailance and earth observation
 \item LEO $\rightarrow$ high spatial resolution
 \item higher temporal resolution is provided by constellations with several satellites in the same orbit
\end{itemize}
