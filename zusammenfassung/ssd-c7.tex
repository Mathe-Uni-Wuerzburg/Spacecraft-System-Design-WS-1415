\section{Rocket Propulsion}

\f{Propulsion systems:} \textit{(experimental)}
\begin{itemize}
 \item chemical: solid, liquid, hybrid, \textit{gelled}
 \item electrical: \textit{electo-thermal, electrostatic}, electromagnetic
 \item photonic: \textit{photon, solar sails}
 \item nuclear: \textit{solid core, gas core, nuclear electric}
 \item cold gas thrusters
\end{itemize}

\f{principals}
\begin{itemize}
 \item ejection of mass, provided by onboad means
 \item conservation of momentum, no momentum transfer to external medium
 \item continuous acceleration
\end{itemize}

\f{staged vehicles}
\begin{itemize}
 \item tandem staged
 \item parallel staged
\end{itemize}

\f{liquid Propulsion systems}
\begin{itemize}
 \item pressure feed system: high-pressure gas supply, pressure regulation, most simple and reliable
 \item turbopump feed system: propellant pressurized by pump, driven by turbine, high thrust and long duration
\end{itemize}

\f{selection criteria}
\begin{itemize}
 \item performance: specific impulse, energy release per propellant mass, combustion, ignition, coolant performance
 \item economic: availability, cost, logistics
 \item handle: condition at ambient, non-toxic, non-corrosive, hazards
\end{itemize}

\f{mono-propellants}
\begin{itemize}
 \item energy release by decomposition, stable under controlled environment
 \item ignition: thermally, catalytic
 \item advantages: simple tankage, feeding, flow, injection
 \item e.G.: hydrogen peroxide ($H_2O_2$), hydrazine ($N_2H_4$)
\end{itemize}

\f{bipropellants}
\begin{itemize}
 \item chemical reaction of two propellants ($O_2, H_2$ or $O_2$, kerosene)
 \item separate storage, mixing
 \item high performance, safe operation
 \item \f{hypergolic propellants:} toxic, trained personel required, pollution risk at launch failure
 \item \f{cryogenic propellants:} gaseous at ambient, need thermal insulation, high power
\end{itemize}

\f{combustion}
\begin{itemize}
 \item before chem. reaction, fuel has to atomize/evaporate
 \item mixing of propellants
 \item timescale: chem $\ll$ atomization, evaporation, mixing
 \item temperatur increase $\rightarrow$ gas volume increase $\rightarrow$ velocity increase
 \item chamber cooling: cooling fluids (fuel), film injection, thermal emission
\end{itemize}

\f{ignition}
\begin{itemize}
 \item pyro: solid propellant, electrilly ignited
 \item spark plug: sparks ignite in combustion chamber
 \item spark torch: seperate igniter combustion chamber
 \item laser: beam focused in combustion chamber
\end{itemize}

\f{solid propellants}
\begin{itemize}
 \item long time storage
 \item range of thrust levels: $2N\dots 10MN$
 \item no moving parts, no service
 \item no shutoff, toxic
 \item applications: boosters, upper stage engines, tactical missiles, gas generation
\end{itemize}

\f{electric propulsion}
\begin{itemize}
 \item electrothermal: heating of propellant by contact with hot metal
 \item electrostatic: acceleration of charged particels 
 \item electromagnetic: acceleration of highly ionized plasma
\end{itemize}

\f{launchers}\\
\begin{tabular}{l|clcc}
 & first launch & space ports& LEO & GTO \\
 \hline
 HII (Japan) & 1994 & Tanegashima& 19 T & 4-8 T\\
 Soyuz (Russland)& 1957 & Baikonur/Plesetsk& 6 T& 1.3 T\\
 Ariane 5 (Europa) & 1996& Kourou& -- & 9.6 T \\
\end{tabular}
