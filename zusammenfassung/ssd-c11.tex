\section{Thermal Testing}
\f{Thermal control tasks}
\begin{itemize}
\item Thermal Engineering
\begin{itemize}
 \item definition of thermal requirements and objectives
 \item establish the thermal design and define the thermal control concept
\end{itemize}
 \item Thermal Analysis
\begin{itemize}
 \item establish a thermal mathematical model TMM
 \item perform temperature distribution calculations
\end{itemize}
 \item Thermal Test
 \begin{itemize}
  \item planning, performance and evaluation of tests under realistic environmental conditions
 \end{itemize}
\end{itemize}

\f{Thermal Control System (TCS) basic definitions}
\begin{itemize}
 \item feat is a form of energy which flows from one body to another body by virtue of the temperature difference, following the temperature gradient
 \item heat transfer takes place via three major mechanisms:
\begin{itemize}
\item radiation: thermal energy transfer via electromagnetic waves
\item convection: thermal energy transfer in a flowing fluid and between the fluid and a solid wall
\item conduction: thermal energy transfer in fluids and solids in the absence of fluid motion
\end{itemize}
\end{itemize}

\f{TCS tasks}
\begin{itemize}
\item the TCS shall conform to the performance requirements (as applicable to the actual project) during all specified mission phases.
\item these mission phases shall be represented by a coherent set of thermal design cases to be proposed by the TCS, covering the extreme range of conditions experienced by an item 
during its lifetime.
\item as a minimum, a hot and a cold worst case shall be defined
\end{itemize}

\f{Thermal Cycling Test}
\begin{itemize}
 \item demonstration that the spacecraft thermal control system is properly designed (including all interfaces to other subsystems)
 \item functional tests of all equipment operating at extreme temperature levels, including an adequate margin (5K to 10K).
 \item verification of the temperature stability during the dedicated thermoelastic stability test phases
 \item final flight temperature level adjustment by trimming of the radiator areas (after TB/TV test) and trimming the heater set points (during/after TB/TV test)
 \item detection of material and workmanship defects by subjecting the equipment to a thermal vacuum environment and extreme temperatures (cycling between worst cold and worst hot 
 temperature)
\end{itemize}

\f{Thermal Balance Test}
\begin{itemize}
 \item verification of the Thermal-Mathematical-Model (TMM) at two worst case hot/cold steady-state conditions (TB-phases) during test phases which simulate closely the flight environment
 \item test results are used to correlate the TMM (precondition for accurate thermal flight prediction)
\end{itemize}

\f{Summary}
\begin{itemize}
 \item start preparing the thermal test well in advance
 \item think twice and discuss all test steps with the other subsystems and the system engineering manager
 \item establish a clear and unambiguous “step-by-step procedure”
 \item think about any potential emergencies in advance and establish “what-if” plans
 \item use reliable test equipment
 \item better too much than too less test data
 \item check the whole system out in test ready configuration but with open chamber door
 \item document every step of the test in the “as-run-procedure”
 \item start immediately with the test evaluation
\end{itemize}