\documentclass[a4paper,12pt]{article}
%\documentclass[a4paper,10pt]{scrartcl}

\usepackage[utf8]{inputenc}
\usepackage[english]{babel}
\usepackage[pdftex]{graphicx}
\usepackage{amssymb}
\usepackage{marvosym}
\usepackage{amsmath}
\usepackage{array}
\usepackage{upgreek}
\usepackage{geometry}

\geometry{verbose,tmargin=1cm,headheight=80pt,lmargin=2cm,bmargin=4cm,rmargin=2.5cm}

\title{Exercise 10 - Solution}
\author{}
\date{}


\pdfinfo{%
  /Title    ()
  /Author   ()
  /Creator  ()
  /Producer ()
  /Subject  ()
  /Keywords ()
}

\begin{document}
\maketitle

\section*{Task 10.1}
\begin{enumerate}
 \item The Fermi energy $E_F$ is the energy difference between the conduction band and the valence band at 0 Kelvin. An electron in the valence band can jump to the 
 conduction band by gaining an amount of energy equal to $E_F$.
 \item given: $T=300K$, $E = E_F + 0.1eV$, $k = 8.617343\cdot 10^{-5}\frac{eV}{K}$
 \[f(E) = \frac{1}{1+e^{\left(\frac{E-E_F}{kT}\right)}} = \frac{1}{1+e^{\left(\frac{E_F + 0.1eV - E_F}{kT}\right)}} = 
 \frac{1}{1+e^{\left(\frac{0.1eV}{kT}\right)}} = 0.020468879 \approx 2\%\]
 \item given: $T'=600K$ $E = E_F + 0.1eV$, $k = 8.617343\cdot 10^{-5}\frac{eV}{K}$
 \[f(E) = \frac{1}{1+e^{\left(\frac{E-E_F}{kT'}\right)}} = \frac{1}{1+e^{\left(\frac{E_F + 0.1eV - E_F}{kT'}\right)}} = 
 \frac{1}{1+e^{\left(\frac{0.1eV}{kT'}\right)}} = 0.126299213 \approx 12.6\%\]
 The probability, that an energy level E contains an electron, increased. This is due to higher temperature which leads to higher thermal energies of 
 electrons which results in a higher possiblities that electrons make their way through the band gap into the conduction band. 
\end{enumerate}

\section*{Task 10.2}
 \begin{enumerate}
  \item BOL = used to determine the condition of states of the spacecraft over its lifetime in order to take into account that the component experiences 
  degradation
  \item EOL = power requirements at end of life are important to guarantee the successful completion of the mission. Oversizing at BOL results in enough 
  power at EOL
 \end{enumerate}

\begin{enumerate}
 \item  BOL = conditions of spacecraft component at the beginning of life, EOL = power requirements at end of life.
 \item  Since solar cells and batteries have limited lifetimes and a performance degradation, the design of a satellite must account for BOL and EOL. 
\end{enumerate}

\section*{Task 10.3}
\begin{enumerate}
 \item \[P_{solArr} = \frac{\frac{\left(\text{power required in eclipse}\right)\cdot \left(\text{time in eclipse}\right)}{\text{path efficiency in 
 eclipse}} + \frac{\left(\text{power required in daylight}\right)\cdot \left(\text{time in daylight}\right)}{\text{path efficiency in 
 daylight}}}{\text{time in daylight}} = \]
\[ = \frac{\frac{100W\cdot20min}{0.65} + \frac{330W\cdot68min}{0.85}}{68min} = 438.01W\]
 \item \[P_{BOL} = P_o \cdot I_d \cdot cos(\Theta) = 253\frac{W}{m^2}\cdot 0.77\cdot cos(23.5^{\circ}) = 178.65\frac{W}{m^2}\]
 \item calculating new $I_d$: \[I_d = (1-0.0275)^5 = 0.87\] 
 \[P_{EOL} = P_{BOL} \cdot I_d = 155.4\frac{W}{m^2}\]
 \item \[A_{solArr} = \frac{P_{solArr}}{P_{EOL}} = \frac{438.01W}{155.4W}m^2 = 2.82m^2\]
 \item \[C_{\text{battery}} = \frac{\left(\text{power required in eclipse}\right)\cdot \left(\text{time in eclipse}\right)}{\left(\text{depth of 
 discharge}\right)\cdot\left(\text{transmission efficiency}\right)} =\]
 \[ = \frac{100W\cdot 22min}{0.2\cdot 0.9} = 203.70Wh\]
 convert to Ah: 
 \[C_{\text{battery}}[Ah] = \frac{C_{\text{battery}}}{U} = \frac{203.70Wh}{27.1V} = 7.52Ah\]
\end{enumerate}

\end{document}
