\documentclass[a4paper,12pt]{article}
%\documentclass[a4paper,10pt]{scrartcl}

\usepackage[utf8]{inputenc}
\usepackage[english]{babel}
\usepackage[pdftex]{graphicx}
\usepackage{amssymb}
\usepackage{marvosym}
\usepackage{amsmath}
\usepackage{array}
\usepackage{upgreek}
\usepackage{geometry}

\geometry{verbose,tmargin=1cm,headheight=80pt,lmargin=2cm,bmargin=4cm,rmargin=2.5cm}

\title{Exercise 8 - Solution}
\author{}
\date{}

\pdfinfo{%
  /Title    ()
  /Author   ()
  /Creator  ()
  /Producer ()
  /Subject  ()
  /Keywords ()
}

\begin{document}
\maketitle

\section*{Task 8.1}
series conduction: \[\sigma_{tot} = \frac{1}{\frac{1}{\sigma_1}+\frac{1}{\sigma_2} + \frac{1}{\sigma_3}}\]
\[\sigma_1 = \frac{L_1}{K_1A_c}, \sigma_2 = \frac{1}{\alpha_cA_c}, \sigma_3 = \frac{L_2}{K_2A_{eff}}\]
\[\Rightarrow \sigma_{tot} = 0.834 \frac{W}{K}\]
\[Q = \sigma_{tot}(T_{box}-T_2) \Rightarrow T_{Box} = \frac{Q}{\sigma_{tot}}+T_2 = 30.02^{\circ}C\]

\section*{Task 8.2}
given: $f=2GHz$, $r=\frac{880}{221}$, $v=10\frac{km}{s}$, $d_{rate}=10\frac{kbit}{s}$, $c=3\cdot 10^8\frac{m}{s}$, $f_c=150kHz$ 

\subsection*{1.}
\begin{itemize}
 \item distance is increasing: 
 \[ \Delta f_{up} = -\frac{f \cdot v}{c} = -66.67kHz\]
 \item distance is decreasing: 
 \[\Delta f_{down} = -\frac{(f+f_{up}) \cdot v \cdot r}{c} = -265.45kHz\]
\end{itemize}

\subsection*{2.}
\begin{itemize}
 \item Doppler shift on received downlink subcarrier:
 \[\Delta f_{sub} = -\frac{v\cdot f_c}{c} = -5Hz\]
 \item Doppler shift on the TM data rate:
 \[\Delta f_{dop} = -\frac{v \cdot d_{rate}}{c} = -0.33\frac{bit}{s}\]
\end{itemize}

\subsection*{3.}
The shape of the transmitted rectangular TM bit stream does not change, it is still rectangular. However, the frequency changes. 

\section*{Task 8.3}
\subsection*{1.}
The difference between high gain and low gain antenna is the respective beam width. A high gain antenna has a focused, narrow radiowave beam
width, while a low gain antenna has a broad radiowave beam width. 
\subsection*{2.}
A satellite has both a high gain and a low gain antenna for failure reasons. Normally the high gain antenna is used for the primary link 
while the low gain antenna is used as a backup.
\subsection*{3.}
Low gain antennas should be distributed on the space craft in a way, that they cover as many directions as possible. If for example a 
spacecraft has only two low gain antennas, they should be placed on opposite sides of the spacecraft. 


\end{document}
