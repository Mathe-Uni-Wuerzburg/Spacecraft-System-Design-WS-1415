\begin{chapter}{17.10.2014}
 

\textbf{Keplerian Orbit Transfer}\\
 Calculate a transfer between two circular orbits with radius $r_A$ to $r_B$. The velocity at pericenter of the transfer ellipse:
 \[ v_p^2 = 2 \mu \left( \frac{1}{r_A} - \frac{1}{r_A + r_B} \right) = 2 \mu \frac{r_B}{r_A(r_A+r_B)} \]
 The required $\Delta v_A$ to inject from the transfer orbit:
 \[ \Delta v_A = v_P - v_A = \sqrt{\frac{\mu}{r_A}} \left( \sqrt{\frac{2r_B}{r_A+r_B}} -1\right) \]
 The required $\Delta v$ to inject from the transfer orbit into orbit with $r_B$:
 \[ \Delta v_B = v_B - v_\text{apo} = \sqrt{\frac{\mu}{r_B}} \left( 1 - \sqrt{\frac{2r_A}{r_A+r_B}}\right) \]
 where $v_B$ is the circular velocity at $r_B$.
 
 The \indexx{Hohmann transfer} is the most energy-efficient transfer between two circular orbits.
 \[ \Delta v_\text{total} =  \Delta v_A + \Delta v_B = \sqrt{\mu}\left[ \sqrt{\left( \frac{2}{r_A} - \frac{2}{r_A+r_B} \right)} - \sqrt{\frac{1}{r_A}} + \sqrt{\frac{2}{r_B}-\frac{2}{r_A+r_B}} - \sqrt{\frac{1}{r_B}} \right]\]

 \setcounter{chapter}{2}
 \setcounter{section}{2}
 \begin{section}{Mission Analysis}
 Selection of orbit parameters for most appropriate orbits for the given task.
 
 \textbf{\indexx{Earth synchronous orbits}}: after a given period of time the subsatellite's ground track repeats. \indexx{Rotation of Earth}: $\tau_E$ \indexx{sidereal day} ($ = 86161.10555 + 0.015C [s]$ where $C$ is the number of centuries after 2000). As Earth rotation is in eastward direction, the satellite orbit with respect to Earch seems to have west direction.
 \[ \Delta \Phi_R = \frac{\tau}{\tau_E}-2\pi \radrev \]
 \begin{tabular}{c|l|c}
  $\tau$ & satellite orbit period& 
 \end{tabular}

 $J_2$ theory taking into account Earth oblateness, we can calculate the satellite orbit plane rotation:
 \[\Delta \Omega = \frac{3\pi J_2 R_E^2 \cos i}{a^2(1-\varepsilon^2)^2} ~~~[\text{rad}/\text{rev}]\]
 total displacement at equator: $\Delta \Phi = \Delta \Phi_R - \Delta \Omega$.\\
 For Earth synchronous orbits, the following property needs to be satisfied:
 \[ n\Delta\Phi = m2\pi \]
 \begin{tabular}{c|l|c}
  $n$ & number of orbits before identical ground track repeats & \\
  $m$ & number of Earth revolutions (day) before identical ground track repeats & 
 \end{tabular}
 
 \textbf{\indexx{Sun synchronous orbits}}: \\
 the sunlight incidents repats\\
 take into account the motion of Earth around Sun. Due to the Earth's orbit around Sun, the Earth/Sun-line intersect at points $P$ moving westward.
 \[ \theta = \frac{360\degree}{365.25 ~\text{days}} \approx 1 ~~~[\degree/\text{day}]\]
 orbital period of Earth around Sun $\tau_{ES}$: $3.155815 \cdot 10^7 ~s \approx 365.25 ~\text{days}$.\\
 angular velocity $\theta = 2 \pi \frac{\tau_E}{\tau_{ES}}~~~[\text{rad}/\text{day}] = 2\pi \frac{\tau_E}{\tau_{ES}}\cdot \frac{\tau}{\tau_E}\radrev$\\
 and for Sun synchronous orbits $\Leftrightarrow \Delta \Omega = \theta$.\\
 Sun and Earth synchronous orbit:
 \[ n(\Delta\Phi_R - \Delta\Omega) = m2\pi \Rightarrow n\tau\left(1-\frac{\tau_E}{\tau_{ES}}\right) = m\tau_E \]
 displacement between subsequent orbits ($\rightarrow$ gap in observation areas): 
 \[ \Delta\Phi = 2\pi\tau\left(\frac{1}{\tau_E}-\frac{1}{\tau_{ES}}\right) = 7.27\cdot10^{-5}\cdot \tau\radrev \]
 typical Earth observation orbits: $550 - 750~km$ above surface $\Rightarrow \Delta\Phi \approx 4.3\cdot 10^{-1} ~\text{rads}$ and offset $\approx 2875~km$.
 
 Sun and earth synchronous satellites \index{sun-synchronous orbits}:\\
 \begin{tabular}{ccc}
  m & n & corresponding altitude\\
  1&14&$894 km$\\
  1&13&$567 km$\\
  1&12&$275 km$\\
  2&22&\\
 \end{tabular}
 
 minimum drift orbits: $\frac{n \pm 1}{m} = k$, $k \in \mathbb{N}$, integer. orbits of successive days to fill the displacements between two successive orbits.\\
 \begin{tabular}{c|l|c}
  $k$ & orbits of successive days to fill the displacements between two successive orbits& 
 \end{tabular}
 
 \underline{Example:} Landsat 1/2\\
 18 day repeat period minimum drift orbit
 \[ \tau \approx 103.3\text{min, } i=99\degree \text{, } \varepsilon = 0.002\text{, apogee}=920km\text{, descending node: } 9:38 \text{local time}\]
 $\Rightarrow$ orbit seperation between successive orbits $\approx 2875km$.\\
 $m=18 \Rightarrow$ distance between adjacent ground tracks $=160km$.
 
 \textbf{\indexx{Eclipse calculation}}:\\
 Angle between earth-sun direction $\vec{s}$ and the orbit plane $=\beta$.
 \[ \beta = \sin^{-1}(\vec{s}\vec{n}) \]
 \begin{tabular}{c|l|c}
  $\vec{n}$ & normal to orbit plane & 
 \end{tabular}
 \[\sin \beta^* = \frac{r_E}{r_E +h} \Rightarrow \beta^* = \sin^{-1}\left(\frac{r_E}{r_E +h}\right) \]
 \[ E_1E_2 = 2\Delta v \text{   eclipse orbital arc}\]
 \[\Delta v = \cos^{-1}\left(\frac{\cos \beta^*}{\cos \beta}\right)\text{   in triangle }ACE_1 \]
 eclipse fraction of a circular orbit:
 \[ F_e = \frac{2\Delta v}{2\pi} = \frac{1}{\pi}\frac{\sqrt{h^2+2r_E h}}{(r_E+h)\cos\beta} \text{   [rad]}\]
 \begin{tabular}{c|l|c}
  $h$ & altitude & \\
  $\beta$ & angle sun-orbit-plane& \\
  $r_E$ & earth radius & 
 \end{tabular}
 
  \indexx{launch energy}: $\varepsilon = \cos^{-1}(\cos^2 i + \sin^2 i -\cos \Delta \Omega)$\\
  change angle related to plane (proportional):\\
  $\Delta \Omega$: longitudal seperation\\
  launch window: $2\frac{\Delta\Omega}{w_c} \Rightarrow \Delta\Omega = w_c t$
  \begin{tabular}{c|l|c}
   $w_c$ & earth rotation rate & 
  \end{tabular}

  \textbf{ground contact and coverage analysis}
  \[\sin \rho = \cos \lambda_0 = \frac{r_E}{r_E+h}\]
  from known $\lambda$ (from $\Lambda_t, \theta_t / \Lambda_s, \theta_s$) we derive $\eta$.
  \[ \tan\eta = \frac{\sin\rho\sin\lambda}{1-\sin\rho\sin\lambda} \]
  \[ \cos\varepsilon = \frac{\sin\eta}{\sin\rho} \Rightarrow \sin\eta = \cos\varepsilon\sin\rho\]
  \[ \eta+\varepsilon+\lambda = 90\degree \]
  \[D =r_E \frac{\sin\lambda}{\sin\eta}\]
  
  \begin{enumerate}[1)]
 \item angular Radius of Earth $\varphi$: $\sin \varphi = \frac{R_E}{R_E+h}$
 \item compute spacecraft viewing angles from subsatellite point $(\Lambda_s,\theta_s)$ and target point $(\Lambda_t,\theta_t)$:
 \[\cos \lambda = \sin\theta_s \sin\theta_t+\cos\theta_s \cos\theta_t \cos |\Lambda_s-\Lambda_t|\]
 \[\cos A_z = \frac{\sin\theta_t-\cos\lambda\sin\theta_s}{\sin\lambda\cos\theta_s} ~~~\text{\indexx{Azimuth}}\]
 \[\tan \eta = \frac{\sin \varphi \sin \lambda}{1-\sin \varphi \cos \lambda}\]
 \item compute coordinates on Earth:
 \[\cos \varepsilon = \frac{\sin\eta}{\sin\varphi}\]
 \[\lambda + \eta+\varepsilon = 90 \degree\]
\end{enumerate}
  \end{section}
\end{chapter}